%
%       Shantanu Yadav
%       September 15, 2020
%-------------------------------------------------------------------------------
\documentclass[11pt]{article}
\textheight=9.5in
\textwidth=6.5in
\oddsidemargin=+10mm
\evensidemargin=+6mm
\topmargin=-0.8in

\usepackage{enumerate,float}
\usepackage{fancyhdr,amssymb,lastpage}
\usepackage{amsmath}
\usepackage[colorlinks]{hyperref}
\usepackage{tikz}
\renewcommand{\rmdefault}{pnc}
\makeatletter
\newcommand{\LeftEqNo}{\let\veqno\@@leqno}
%------------------------------------------------------------------------------
%--------------------------------- B E G I N   T H E   D O C U M E N T --------
\begin{document}
%--------------------------------- Title of Document 
%((((((------------------------------------------------------------------------
%\baselineskip=18pt
 \begin{center}
         {\LARGE \bf
	 EE : Convex Optimization \\
             Lab 6 Project Formulation
         }
 \end{center}
 \vspace{1ex}
 \begin{center}
	 {\Large \bf Shantanu Yadav}, \quad {\bf EE20MTECH12001}
 \end{center}
 \begin{center}
	 \hrule
 \end{center}
 \vspace{1ex}
%))))))------------------------------------------------------------------------

%------------------------------------------------------------------------------
%\section{Project Formulation}
%Accessories and co. is producing three kinds of covers for Apple products: one for iPod, one for iPad, and one for iPhone. The company's production facilities are such that if we devote the entire production to iPod covers, we can produce 6000 of them in one day. If we devote the entire production to iPhone covers or iPad covers, we can produce 5000 or 3000 of them in one day. The production schedule is one week (5 working days), and the week's production must be stored before distribution. Storing 1000 iPod covers (packaging included) takes up 40 cubic feet of space. Storing 1000 iPhone covers (packaging included) takes up 45 cubic feet of space, and storing 1000 iPad covers (packaging included) takes up 210 cubic feet of space. The total storage space available is 6000 cubic feet. Due to commercial agreements with Apple, Accessories & co. has to deliver at least 5000 iPod covers and 4000 iPad covers per week in order to strengthen the product's diffusion. The marketing department estimates that the weekly demand for iPod covers, iPhone, and iPad covers does not exceed 10000 and 15000 , and 8000 units, therefore the company does not want to produce more than these amounts for iPod, iPhone, and iPad covers. Finally, the net profit per each iPod cover, iPhone cover, and iPad cover is 4,6 and 10 dollars respectively. The aim is to determine a weekly production schedule that maximizes the total net profit.
%\begin{enumerate}[{1}{$a$):]
%	\item
%
%		Write a Linear Programming formulation for the problem. Start by stating any assumptions that you make. Label each constraint. For this first formulation, the decision variables should represent the proportion of time spent each day on producing of the two items:%\\
%		$x_1 =$ proportion of time devoted each day to iPod cover production\\
%		$x_2 =$ proportion of time devoted each day to iPhone cover production\\
%		$x_3 =$ proportion of time devoted each day to iPad cover production
%\end{enumerate}
%\end{document}
%
