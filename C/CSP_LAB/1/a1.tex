\documentclass[journal,12pt,twocolumn]{IEEEtran}

\usepackage{setspace}
\usepackage{gensymb}

\singlespacing


\usepackage[cmex10]{amsmath}

\usepackage{amsthm}

\usepackage{mathrsfs}
\usepackage{txfonts}
\usepackage{stfloats}
\usepackage{bm}
\usepackage{cite}
\usepackage{cases}
\usepackage{subfig}

\usepackage{longtable}
\usepackage{multirow}

\usepackage{enumitem}
\usepackage{mathtools}
%\usepackage{steinmetz}
\usepackage{tikz}
\usepackage{circuitikz}
\usepackage{verbatim}
%\usepackage{tfrupee}
\usepackage[breaklinks=true]{hyperref}

\usepackage{tkz-euclide}

\usetikzlibrary{calc,math}
\usepackage{listings}
    \usepackage{color}                                            %%
    \usepackage{array}                                            %%
    \usepackage{longtable}                                        %%
    \usepackage{calc}                                             %%
    \usepackage{multirow}                                         %%
    \usepackage{hhline}                                           %%
    \usepackage{ifthen}                                           %%
    \usepackage{lscape}     
\usepackage{multicol}
\usepackage{chngcntr}

\DeclareMathOperator*{\Res}{Res}

\renewcommand\thesection{\arabic{section}}
\renewcommand\thesubsection{\thesection.\arabic{subsection}}
\renewcommand\thesubsubsection{\thesubsection.\arabic{subsubsection}}

\renewcommand\thesectiondis{\arabic{section}}
\renewcommand\thesubsectiondis{\thesectiondis.\arabic{subsection}}
\renewcommand\thesubsubsectiondis{\thesubsectiondis.\arabic{subsubsection}}


\hyphenation{op-tical net-works semi-conduc-tor}
\def\inputGnumericTable{}                                 %%

\lstset{
%language=C,
frame=single, 
breaklines=true,
columns=fullflexible
}
\begin{document}


\newtheorem{theorem}{Theorem}[section]
\newtheorem{problem}{Problem}
\newtheorem{proposition}{Proposition}[section]
\newtheorem{lemma}{Lemma}[section]
\newtheorem{corollary}[theorem]{Corollary}
\newtheorem{example}{Example}[section]
\newtheorem{definition}[problem]{Definition}

\newcommand{\BEQA}{\begin{eqnarray}}
\newcommand{\EEQA}{\end{eqnarray}}
\newcommand{\define}{\stackrel{\triangle}{=}}
\bibliographystyle{IEEEtran}
\providecommand{\mbf}{\mathbf}
\providecommand{\pr}[1]{\ensuremath{\Pr\left(#1\right)}}
\providecommand{\qfunc}[1]{\ensuremath{Q\left(#1\right)}}
\providecommand{\sbrak}[1]{\ensuremath{{}\left[#1\right]}}
\providecommand{\lsbrak}[1]{\ensuremath{{}\left[#1\right.}}
\providecommand{\rsbrak}[1]{\ensuremath{{}\left.#1\right]}}
\providecommand{\brak}[1]{\ensuremath{\left(#1\right)}}
\providecommand{\lbrak}[1]{\ensuremath{\left(#1\right.}}
\providecommand{\rbrak}[1]{\ensuremath{\left.#1\right)}}
\providecommand{\cbrak}[1]{\ensuremath{\left\{#1\right\}}}
\providecommand{\lcbrak}[1]{\ensuremath{\left\{#1\right.}}
\providecommand{\rcbrak}[1]{\ensuremath{\left.#1\right\}}}
\theoremstyle{remark}
\newtheorem{rem}{Remark}
\newcommand{\sgn}{\mathop{\mathrm{sgn}}}
\providecommand{\abs}[1]{\left\vert#1\right\vert}
\providecommand{\res}[1]{\Res\displaylimits_{#1}} 
\providecommand{\norm}[1]{\left\lVert#1\right\rVert}
%\providecommand{\norm}[1]{\lVert#1\rVert}
\providecommand{\mtx}[1]{\mathbf{#1}}
\providecommand{\mean}[1]{E\left[ #1 \right]}
\providecommand{\fourier}{\overset{\mathcal{F}}{ \rightleftharpoons}}
%\providecommand{\hilbert}{\overset{\mathcal{H}}{ \rightleftharpoons}}
\providecommand{\system}{\overset{\mathcal{H}}{ \longleftrightarrow}}
	%\newcommand{\solution}[2]{\textbf{Solution:}{#1}}
\newcommand{\solution}{\noindent \textbf{Solution: }}
\newcommand{\cosec}{\,\text{cosec}\,}
\providecommand{\dec}[2]{\ensuremath{\overset{#1}{\underset{#2}{\gtrless}}}}
\newcommand{\myvec}[1]{\ensuremath{\begin{pmatrix}#1\end{pmatrix}}}
\newcommand{\mydet}[1]{\ensuremath{\begin{vmatrix}#1\end{vmatrix}}}
\numberwithin{equation}{subsection}
\makeatletter
\@addtoreset{figure}{problem}
\makeatother
\let\StandardTheFigure\thefigure
\let\vec\mathbf
\renewcommand{\thefigure}{\theproblem}
\def\putbox#1#2#3{\makebox[0in][l]{\makebox[#1][l]{}\raisebox{\baselineskip}[0in][0in]{\raisebox{#2}[0in][0in]{#3}}}}
     \def\rightbox#1{\makebox[0in][r]{#1}}
     \def\centbox#1{\makebox[0in]{#1}}
     \def\topbox#1{\raisebox{-\baselineskip}[0in][0in]{#1}}
     \def\midbox#1{\raisebox{-0.5\baselineskip}[0in][0in]{#1}}
\vspace{3cm}
\title{EE5301 ASSIGNMENT 1}
\author{SHANTANU YADAV }
\maketitle
\newpage
\bigskip
\renewcommand{\thefigure}{\theenumi}
\renewcommand{\thetable}{\theenumi}
\section{Problem}
Implement basic signal processing algorithms such as convolution, correlation, downsampling and upsampling 
\section{C Code Implementation}
The C code solution code is available at
\begin{lstlisting}
https://github.com/Shantanu2508/mtech/tree/master/C/CSP_LAB/1
\end{lstlisting}
%
\section{Theory}
\subsection{Convolution}
The complete characterization of a LTI system can be done by its impusle response. Mathematically,
\begin{align*}
	x[n]=\delta[n] \\ 
	y[n]=h[n]
\end{align*}
According to the $\mathbf{shifting}$ $\mathbf{property}$, any signal can be produced as combination of weighted and shifted impulses; that is
\begin{align*}
	x[n]=\sum\limits_{k=\infty}^{\infty} x[k]\delta[n-k] \\
	\implies 
	y[n]=\sum\limits_{k=\infty}^{\infty} x[k] T\cbrak{\delta[n-k]} \\
	y[n]=\sum\limits_{k=\infty}^{\infty} x[k] h[n-k] \\
\end{align*}
The operation can be symbolically as 
\begin{align*}
	y[n]=x[n]*h[n]
\end{align*}
For a causal LTI system the output depends upon past and present values. \\
For a discrete time causal LTI system 
\begin{align*}
	h[n]=0 \text{\quad for \quad} k<0 \\
	y[n]=\sum\limits_{k=0}^{\infty} h[k] x[n-k] \\
\end{align*}
\subsection{Correlation}
Signal correlation can be considered as a measure of similarity of two signals. In signal processing systems, when a signal is corrupted by another undesirable signal (noise) signal estimation is performed by finding the correlation between the corrupted signal and the original signal. \\
Given two discrete-time real signals (sequences) $x[k]$ and $y[k]$ cross-correlation is defined as
\begin{align*}
	R_{xy}[k] = \sum\limits_{n=-\infty}^{\infty} x[n]y[n-k]
\end{align*}
where the parameter $k$ is any integer $-\infty \leq k \leq \infty$ .\\
Correlation can also be evaluated using the discrete-time convolution as follows 
\begin{align*}
	R_{xy}[k] = x[k]*y[-k]
\end{align*}
\subsubsection{Properties of correlation}
1. Signal crosscorrelation can be also considered as a measure of similarity of two signals.\\
2. Signal autocorrelation indicates how the signal energy (power) is distributed within the signal, and as such is used to measure the signal power.\\
3. $R_{xx}[0] = E_{\infty}^x$ \\
4. $R_{xx}[k] \leq R_{xx}[0]$ \\
5. $R_{xy}[k] = R_{yx}{-k}$
\subsection{Up/Down sampling}
\subsubsection{Upsampling}
In upsampling the sampling rate is increased by inserting additional sample points of weight 0 between successive samples.
\begin{align*}
	y[n] = x[\frac{n}{L}]
\end{align*}
\subsubsection{Down sampling} In down sampling the sampling rate is decreased by selecting every alternate M samples.
\begin{align*}
        y[n] = x[Mn]
\end{align*}
\end{document}
